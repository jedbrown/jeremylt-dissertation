When the PDE in Equation \ref{eq:weak_form} is linear, the pointwise functions $f_0$ and $f_1$ are also linear and Krylov subspace methods can be used to solve the Galerkin system of equations in Equation \ref{eq:galerkin_form}.
When the PDE is non-linear, the Galerkin system in Equation \ref{eq:galerkin_form} provides the residual evaluator for a non-linear solver and Jacobian of this non-linear residual can be represented in a similar fashion as Equation \ref{eq:galerkin_form}, based upon the weak form
\begin{equation}
\langle v, J \left( u \right) w \rangle = \int_{\Omega}
\left[ \begin{array}{c c}
v^T & \left( \nabla v \right)^T
\end{array} \right]
\left[ \begin{array}{c c}
f_{0, 0} & f_{0, 1}\\
f_{1, 0} & f_{1, 1}
\end{array} \right]
\left[ \begin{array}{c}
w \\ \nabla w
\end{array} \right]
\label{eq:jacobian_form}
\end{equation}
where $f_{i, 0} = \frac{\partial f_i}{\partial u} \left( u, \nabla u \right)$ and $f_{i, 1} = \frac{\partial f_i}{\partial \nabla u} \left( u, \nabla u \right)$.
If these pointwise functions $f_{i, j}$ are not available analytically, they can be computed via algorithmic differentiation or finite differencing.

With these pointwise functions, we can describe Jacobian-free Newton-Krylov methods for solving non-linear PDEs.
Jacobian-free Newton-Krylov methods were summarized, with preconditioning strategies, by Knoll and Keyes in \cite{knoll2004jacobian}.