This framework for LFA of arbitrary order finite element discretizations has several opportunities for continued research.

The work in this dissertation focused on LFA of nodal finite element discretizations where all fields used the same discretization.
One natural extension of this work is the LFA of PDEs discretized with mixed finite element methods, such as those seen in solid mechanics applications with a continuous displacement space and a discontinuous pressure space.
Another natural extension is the LFA of PDE with modal or hierarchical bases, where the basis is not associated with nodal locations.

We described the LFA of $p$-multigrid and $h$-multigrid methods.
This framework can also analyze $hp$-multigrid methods, where the prolongation operator is again defined by the evaluation of the coarse grid basis on the fine grid macro-element space.
Exploration of these techniques with LFA is an area for future work.

Currently, our LFA of BDDC is restricted to primal spaces only consisting of subdomain vertices, but it is common to augment these points with subdomain edge and face averages, especially in three dimensional problems.
Extending the LFA to use richer primal spaces would allow for more general and practical analysis.

Another popular family of domain decomposition method is Schwartz methods, such as additive Schwartz.
Developing the LFA of Schwartz methods, or other overlapping domain decomposition techniques, would help quantify convergence differences between overlapping methods and BDDC and facilitate a broader comparison between these two types of methods in terms of both convergence rates and global communication required.

This LFA framework is designed to support applications.
The results form the LFA in this dissertation indicates two new areas of note for application development.

First, this LFA indicates that using Dirichlet BDDC with single high-order element subdomains is an attractive preconditioner.
Single high-order element subdomains can use the FDM to provide high-order matrix-free subdomain solvers.
Furthermore, using a FDM based subdomain solver can eliminate the increased setup time for Dirichlet BDDC compared to lumped BDDC.
A fully matrix-free BDDC implementation using libCEED and PETSc is in development and will be a novel implementation.

Secondly, this LFA indicates that BDDC has better smoothing properties for $p$-multigrid than Jacobi preconditioned Chebyshev iterations.
This improved smoothing also requires reduced communication, which makes BDDC a very attractive smoother for $p$-multigrid.
BDDC smoothing for $p$-multigrid is also completely novel and development of an example will follow a fully matrix-free BDDC implementation using libCEED and PETSc.
