High-order matrix-free finite element operators offer superior performance on modern high performance computing hardware when compared to assembled sparse matrices, both with respect to the number of floating point operations needed for operator evaluation and the memory transfer needed for a matrix-vector product.
However, high-order matrix-free operators require iterative solvers, such as Krylov subspace methods, but these iterative solvers converge slowly for the ill-conditioned operators that come from high-order discretizations.
Preconditioning techniques can significantly improve the convergence of these iterative solvers for high-order matrix-free finite element operators.
Specifically, $p$-multigrid and domain decomposition methods are particularly well suited for problems on unstructured meshes, but these methods can have parameters that require careful tuning to ensure proper convergence.
Local Fourier Analysis (LFA) of these preconditioners analyzes the frequency modes found in the iterate error following the application of these methods and can provide sharp convergence estimates and parameter tuning while only requiring computation on a single representative element or macro-element patch.

In the remainder of Chapter \ref{ch:Introduction} we provide some brief notes on the reproducibility of the work in this dissertation and provide context for this work in relation to previous work on preconditioning high-order finite element operators.

In Chapter \ref{ch:HighOrderFEM}, we present a representation of arbitrary second-order partial differential equations (PDEs) for high-order finite element discretizations that facilitates matrix-free implementation.
This representation is used by the Center for Efficient Exascale Discretizations to provide performance portable high-order matrix-free implementations of finite element operators for arbitrary second order partial differential equations.

In Chapter \ref{ch:LocalFourierAnalysis}, we develop LFA of arbitrary order finite element operators represented in this fashion and provide examples of analyzing the with weighted Jacobi and the Chebyshev semi-iterative methods, which are commonly used as smoothers in multigrid methods.

In Chapter \ref{ch:MultigridMethods}, this LFA framework is expanded to create the LFA of $p$-multigrid for Continuous Galerkin discretizations.
This LFA of $p$-multigrid is validated with numerical experiments.
Furthermore, we extend this LFA to reproduce previous work with $h$-multigrid by using macro-elements consisting of multiple low-order finite elements.

In Chapter \ref{ch:DomainDecomposition}, we develop the LFA of the lumped and Dirichlet versions of Balancing Domain Decomposition by Constraints (BDDC) preconditioners.
By using Fast Diagonalization Method approximate subdomain solvers, the increased setup costs for the Dirichlet BDDC preconditioner relative to the lumped variant can be substantially reduced, which makes Dirichlet BDDC an attractive preconditioner.
We validate this work against previous numerical experiments on high-order finite elements, and we can exactly reproduce previous work on the LFA of BDDC by using macro-elements consisting of multiple low-order finite elements.

The performance of $p$-multigrid in parallel is partially determined by the number of multigrid levels.
A larger number of multigrid levels requires additional global communication on parallel machines and these intermediate representations are less computationally efficient than the fine grid representation.
However, aggressive coarsening is not supported by traditional polynomial smoothers, which leads to a gap in the error frequency modes targeted by the smoother and the coarse grid solver and causes degraded convergence.
Dirichlet BDDC can be used as a smoother for $p$-multigrid to target a wider range of error mode frequencies than polynomial smoothers such as the Chebyshev semi-iterative method, which facilitates more aggressive coarsening.
We provide LFA of $p$-multigrid with Dirichlet BDDC smoothing to demonstrate the suitability of this combination for preconditioning high-order matrix-free finite element discretizations.

Finally, in Chapter \ref{ch:Conclusions} we provide concluding remarks and areas for future research.