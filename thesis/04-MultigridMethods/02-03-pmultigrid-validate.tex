In this section, we compare the LFA two-grid convergence factors to numerical results.
Our numerical experiments were conducted using the libCEED \cite{libceed-user-manual} with PETSc \cite{petsc-user-ref} multigrid example found in the libCEED repository.
PETSc provides the mesh management, linear solvers, and multigrid preconditioner while libCEED provides the matrix-free operator evaluation.

We recover the manufactured solution given by
\begin{equation}
f \left( x, y, z \right) = x y z \sin \left( \pi x \right) \sin \left( \pi \left( 1.23 + 0.5 y \right) \right) \sin \left( \pi \left( 2.34 + 0.25 z \right) \right)
\end{equation}
on the domain $\left[ -3, 3 \right]^3$ with Dirichlet boundary conditions for finite element discretizations with varying orders on approximately 8 million degrees of freedom for a variety of test cases.
Although LFA is defined on infinite grids, which most naturally translates to periodic problems, LFA-predicted convergence factors are also accurate for appropriate problems with other boundary conditions on finite grids \cite{rodrigo2019validity}.

% -----------------------------------------------------------------------------
% Jacobi Smoothing
% -----------------------------------------------------------------------------

Since the Chebyshev smoothing is based upon the Jacobi preconditioned operator, it is important to validate the LFA of the Jacobi smoothing before considering Chebyshev smoothing.
We use simple Jacobi smoothing with a weight of $\omega = 1.0$ to validate the LFA.

\begin{table}[ht!]
\begin{center}
\begin{tabular}{l c c}
  \toprule
  $p_{\text{fine}}$ to $p_{\text{coarse}}$  &  LFA  &  libCEED  \\
  %\cmidrule(lr){2-3} \cmidrule(lr){4-5} \cmidrule(lr){6-7}
  \toprule
  $p = 2$ to $p = 1$   &  0.312  &  0.301  \\
  \midrule
  $p = 4$ to $p = 2$   &  1.436  &  1.402  \\
  $p = 4$ to $p = 1$   &  1.436  &  1.401  \\
  \midrule
  $p = 8$ to $p = 4$   &  1.989  &  1.885  \\
  $p = 8$ to $p = 2$   &  1.989  &  1.874  \\
  $p = 8$ to $p = 1$   &  1.989  &  1.875  \\
  \bottomrule
\end{tabular}
\end{center}
\caption{LFA and Experimental Two-Grid Convergence Factor with Jacobi Smoothing for 3D Laplacian with $\omega = 1.0$}
\label{table:two_grid_3d_jacobi}
\end{table}

The results in Table \ref{table:two_grid_3d_jacobi} provide the LFA and experimental convergence factors for the test problem.
As expected, the high-order fine grid problems diverge with a smoothing factor of $\omega = 1.0$; however, the LFA provides reasonable upper bounds on the true convergence factor seen in the experimental results.

% -----------------------------------------------------------------------------
% Chebyshev Smoothing
% -----------------------------------------------------------------------------

We used the LFA estimates of the maximal eigenvalue to set the extremal eigenvalues used the Chebyshev iteration in PETSc, using $\lambda_{\text{min}} = 0.1 \hat{\lambda}_{\text{max}}$ and $\lambda_{\text{max}} = 1.0 \hat{\lambda}_{\text{max}}$, where $\hat{\lambda}_{\text{max}}$ is the estimated maximal eigenvalue of the symbol of the Jacobi preconditioned operator.

\begin{table}[ht!]
\begin{center}
\begin{tabular}{l cc cc cc}
  \toprule
  $p_{\text{fine}}$ to $p_{\text{coarse}}$  &  \multicolumn{2}{c}{$k = 2$}  &  \multicolumn{2}{c}{$k = 3$}  &  \multicolumn{2}{c}{$k = 4$}  \\
  %\cmidrule(lr){2-3} \cmidrule(lr){4-5} \cmidrule(lr){6-7}
                      &  LFA  &  libCEED  &  LFA  &  libCEED  &  LFA  &  libCEED  \\
  \toprule
  $p = 2$ to $p = 1$  &  0.253 & 0.222  &  0.076 & 0.058  &  0.041 & 0.033  \\
  \midrule
  $p = 4$ to $p = 2$  &  0.277 & 0.251  &  0.111 & 0.097  &  0.062 & 0.050  \\
  $p = 4$ to $p = 1$  &  0.601 & 0.587  &  0.416 & 0.398  &  0.295 & 0.276  \\
  \midrule
  $p = 8$ to $p = 4$  &  0.398 & 0.391  &  0.197 & 0.195  &  0.121 & 0.110  \\
  $p = 8$ to $p = 2$  &  0.748 & 0.743  &  0.611 & 0.603  &  0.506 & 0.469  \\
  $p = 8$ to $p = 1$  &  0.920 & 0.914  &  0.871 & 0.861  &  0.827 & 0.814  \\
  \bottomrule
\end{tabular}
\end{center}
\caption{LFA and Experimental Two-Grid Convergence Factor with Chebyshev Smoothing for 3D Laplacian}
\label{table:two_grid_3d_chebyshev}
\end{table}

The LFA provides reasonable upper bounds on the true convergence factor seen in the experimental results.
As with the one and two dimensional results, rapid coarsening of the polynomial order of the bases decreases the effectiveness of higher order Chebyshev smoothing.
