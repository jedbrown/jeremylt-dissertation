High-order finite element methods offer accuracy advantages over low-order finite elements \cite{demkowicz1989toward, oden1989toward, rachowicz1989toward} and have spectral convergence properties.
However, high-order finite elements are less common because a linear operator or the Jacobian of a non-linear operator rapidly loses sparsity in a sparse matrix representation.
Matrix-free formulations of these methods \cite{brown2010efficient, deville2002highorder, knoll2004jacobian}, provide efficient implementations of these methods on modern hardware \cite{libceed-user-manual, fischer2020scalability, kronbichler2019multigrid} with better performance than sparse matrix representations, both in terms of storage and floating point operations required to execute a matrix-vector product.

However, high-order discretizations of PDE operators tend to be ill-conditioned, and iterative solvers, such as Krylov subspace methods like the Conjugate Gradient method \cite{hestenes1952methods, shewchuk1994introduction}, are sensitive to the spectrum and condition number of the operator.
Preconditioning these high-order PDE operators is necessary to ensure fast convergence of iterative solvers.

\subsection{Multigrid Methods}

Multigrid methods \cite{brandt1982guide, briggs2000multigrid, stuben1982multigrid} are popular for solving linear systems derived from discretizing PDEs.
These methods can be used as solvers or as preconditioners for other iterative solvers.

Geometric multigrid has become a common component in many high performance computing applications and is available in many packages, such as PETSc.
May et al. \cite{may2016extreme} highlighted several benefits to geometric multigrid in extreme scale computing.
Specifically, geometric multigrid has a scalable setup phase, the ability to use highly performance matrix-free smoothers on all levers, and the ability to use identical stencils (or functions that apply the action of the stencil matrix-free) for all coarse level operators.
Geometric multigrid is therefore seen as "undoubtedly {\textit{the preconditioner of choice}}" for elliptic problems due to this scalability and resolution independent convergence.

$H$-multigrid, which is the typical type of geometric multigrid used for low-order discretizations, coarsens the mesh by aggregating multiple elements.
$P$-multigrid, developed by R{\o}nquist and Patera \cite{ronquist1987spectral}, is instead based on decreasing the order of the bases in high-order or spectral finite element discretizations.

Davydov, et al. \cite{davydov2019matrix} used matrix-free high-order finite element methods for $h$-multigrid preconditioning for hyperelasticity at finite stain, but $p$-multigrid can be a more natural fit for matrix-free high-order methods, especially on unstructured meshes where aggregating fine grid elements to form the coarse grid representation can be somewhat complex.

Algebraic Multigrid (AMG) is based on applying multigrid ideas directly to an assembled matrix representation of the discrete finite element operator.
Heys et al. \cite{heys2005algebraic} investigated AMG for high-order discretizations with some level of success; however this approach relies upon assembling the high-order operator, which discards the performance benefits of a matrix-free representation.
We restrict our use of AMG to the coarse grid solver as part of $p$-multigrid or domain decomposition.
The coarse grid with linear elements has substantially fewer degrees of freedom, by a factor of approximately $p^3 / 2^3$ for $p$-multigrid, which makes assembly and efficient direct solution of the coarse grid far more tractable.

The coarse grid solve on linear elements with AMG can be very efficient, and the application of the operator on the fine grid is very efficient with a matrix-free representation, but intermediate grids between the coarsest and finest levels less less efficient representations.
Furthermore, an increased number of levels requires additional global communication, which can become a bottleneck in parallel applications.
We are therefore interested in geometric multigrid with aggressive $p$-coarsening from the high-order fine grid directly to the linear coarse grid representation.

Polynomial smoothers can be applied matrix-free and offer competitive smoothing properties when compared to Gauss-Seidel \cite{adams2003parallel}.
Furthermore, Brannick et al. \cite{brannick2015polynomial} investigated polynomial smoothers for aggressive coarsening in $h$-multigrid and found Chebyshev adequate in this context; however our analysis indicates that Chebyshev is not an appropriate smoother for aggressive $p$-coarsening.
We need a better smoother for rapid $p$-coarsening directly to linear elements, and we turn to domain decomposition methods to find this improved smoother.

\subsection{Domain Decomposition Methods}

Domain decomposition methods are popular techniques for preconditioning high-order finite element and spectral element discretizations.
Dryja and Widlund surveyed many of the domain decomposition techniques in an effort to unify existing theory \cite{dryja1989towards}.

Schwartz methods for domain decomposition were first developed by Hermann Schwarz \cite{schwarz1972gesammelte} as an alternating method in which solutions in overlapping subregions of the global problem were solved while using previous solutions in neighboring subregions as boundary conditions for the current subregion problem.
In this original formulation, the algorithm is a multiplicative Schwarz method.

Dryja and Widlund \cite{widlund1987additive,dryja1989additive} developed the additive variant of the Schwarz method.
This method is an iterative technique in which the solution is a sum of the projections into the overlapping subspaces, and this modification allows the additive Schwarz method to be used as a preconditioner for the Conjugate Gradient method.

Fischer and Lottes \cite{fischer1997overlapping,fischer2005hybrid} used overlapping additive Schwarz as a preconditioner for the Poisson pressure solve in the spectral element formulation of the incompressible Navier-Stokes equations.
In this work, the overlap between subdomains was successfully reduced to one or two nodes per subdomain; however, $\left( p + 2 \right)^3 - p^3$ or $\left( p + 1 \right)^3 - p^3$, where $p$ is the polynomial order of the high-order or spectral element basis, can be a rather substantial number of nodes that require additional communication across subdomains for the additive Schwarz method.
For this reason, we investigate non-overlapping domain decomposition techniques.

Specifically, we are interested in the Balancing Domain Decomposition By Constraints (BDDC) method.
Dohrmann \cite{dohrmann2003preconditioner} developed BDDC in 2003 as a non-overlapping domain decomposition technique where an energy minimization problem is solved across subdomain boundaries.
This technique was developed as a modification to the earlier Balancing Domain Decomposition (BDD) by Mandel \cite{mandel1993balancing} to help overcome the limitations in BDD when addressing higher order problems, such as fourth-order problems on plates.

BDDC is closely related to the Dual Primal version of Finite Element Tearing and Interconnecting (FETI-DP) developed by Dohrmann \cite{farhat1991method}.
BDDC is formulated in terms of the energy minimization problem across subdomain interfaces rather than in terms of Lagrange multipliers, as seen in FETI-DP.
Connections have been established between FETI techniques and Neumann-Neumann domain decomposition methods \cite{klawonn2001feti}, and operators preconditioned with BDDC and FETI-DP have been shown to have the same spectrum, under certain assumptions \cite{mandel2007bddc}.
In this work, the 'lumped' and 'Dirichlet' nomenclature from FETI-DP is used to describe two variants of the BDDC algorithm.
The lumped BDDC variant employs a naive injection operator into the subassembled problem space with is cheaper to setup and apply while the Dirichlet BDDC variant create a harmonic extension for the jump in values across the broken subdomain interface which ultimately better controls the condition number of the preconditioned operator.

Fischer, Miller, and Tufo \cite{fischer2000overlapping} paired this overlapping additive Schwarz preconditioner with a fast subdomain solver based upon the Fast Diagonalization Method (FDM) by Lynch \cite{lynch1964direct}.
The use of inexact subdomain solvers for BDDC was explored by Li and Widlund \cite{li2007use}.
In this work, multigrid v-cycles were used to solve the partially assembled subdomain problems.

We explore the use of FDM based approximate subdomain solvers for BDDC with subdomains consisting of single high-order finite elements.
This approach was successfully employed by Pavarino, Widlund, and Zampini for nearly incompressible elasticity problems in three dimensions \cite{pavarino2010bddc}.

With finite elements using high-order tensor product discretizations, the construction of these FDM approximate subdomain solvers should help reduce the setup cost for Dirichlet BDDC when compared to lumped BDDC, as solver ingredients can be reused between the subassembled problem solver and the subdomain interior solver used in the harmonic extension problem.
Furthermore, this problem should be cheaper to set up and factoring the dense matrix representation of these problems, as the FDM approximate subdomain solvers are based around simultaneous diagonalization of the one dimensional mass and stiffness problems.

BDDC can be used as a preconditioner for an iterative solver; however we are interested in combining BDDC with p-multigrid.
As mentioned above, multigrid methods can provide effective subdomain solvers for BDDC, but using BDDC as a smoother for multigrid methods is novel.
BDDC has improved smoothing properties when compared to polynomial smoothers, which facilitates rapid coarsening in p-multigrid.
Additionally, BDDC has reduced communication requirements when compared to polynomial smoothers that deliver comparable smoothing properties for more conservative p-coarsening.

The combination of BDDC as a smoother for p-multigrid, especially with aggressive coarsening strategies, is novel work.

\subsection{Local Fourier Analysis}

Local Fourier Analysis (LFA) was developed by Brandt \cite{brandt1977multi,wienands2004practical} as a powerful tool for predicting multigrid performance and tuning of multigrid components by examining the spectral radius and/or condition number of the symbol of the underlying discretized operator, which enables sharp predictions for large-scale problems.
This analysis has subsequently been extended to finite element discretizations and a wide variety of preconditioning methods.
LFA is based on the finite difference stencil or the single element operator matrix for finite element discretizations, which allows accurate parameter tuning for preconditioning techniques with a relatively small amount of computation when compared to the size of the true global problem.

We create a new LFA framework for representing LFA of arbitrary second-order PDEs discretized with high-order finite elements, where the PDE has an arbitrary number of components and the problem is in any number of dimensions.

We develop LFA of p-multigrid with arbitrary second-order PDEs using high-order finite element discretizations.
Although van der Vegt and Rhebergen \cite{van2011discrete} discussed LFA of p-multigrid for the discontinuous Galerkin method, this formulation cannot be extended to the continuous Galerkin method.
Our LFA of p-multigrid on high-order finite element discretizations for the continuous Galerkin method is novel work.

LFA of h-multigrid with high-order finite elements was developed in \cite{he2020two} for Lagrange bases with uniformly spaced nodes.
We extend this framework to reproduce this previous work on the LFA of h-multigrid for high-order finite element discretizations.
Furthermore, our LFA framework can reproduce LFA of finite difference discretizations if the finite difference stencil can be represented as a finite element discretization.

We also develop LFA of lumped and Dirichlet BDDC with arbitrary second-order PDEs using subdomains consisting of single high-order elements with exact and inexact subdomain solvers.
The LFA of lumped and Dirichlet BDDC with exact subdomain solvers was introduced by Brown, He, and MacLachlan \cite{brown2019local} for subdomains consisting of multiple low-order finite elements, and we extend our framework to reproduce this previous work.

This new LFA of p-multigrid and LFA of BDDC on single high-order element subdomains is combined to provide estimates for p-multigrid with BDDC smoothing.

This work is available in an open source Julia package, LFAToolkit.jl \cite{thompson2021toolkit}.
Several software packages have been developed for LFA of h-multigrid methods \cite{rittich2018extending,kahl2020automated,wienands2004practical}, however, to the best of our knowledge, no packages support LFA of p-multigrid methods, especially with arbitrary PDEs.
We are similarly unaware of packages that provide LFA of BDDC, for high or low-order element subdomains.
