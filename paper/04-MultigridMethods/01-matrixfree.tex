Both $h$ and $p$ multigrid follow the same general algorithm, with differences in the grid transfer operators and coarse grid representation.
Application of multigrid follows the algorithm \ref{def:multigrid_algorithm}, similar to the presentation in \cite{brandt1982guide}.
\begin{definition}[Multigrid Algorithm]\label{def:multigrid_algorithm}
~\\
\begin{algorithmic}[1]
\State Compute $\mathbf{u}_k$
\State $\mathbf{u}_k \gets \mathbf{u}_k + \hat{\mathbf{M}}^{-1} \left( \mathbf{b} - {\color{burgundy}\mathbf{A}}_f \mathbf{u}_k \right)$ \Comment{pre-smooth $\nu$ times}
\State $\mathbf{r} = {\color{burgundy}\mathbf{R}}_{\text{ftoc}} \left( \mathbf{b} - {\color{burgundy}\mathbf{A}}_f \mathbf{u}_k \right)$ \Comment{restrict the residual}
\State ${\color{burgundy}\mathbf{A}}_c \mathbf{e} = \mathbf{r}$                                                                          \Comment{Solve on coarse grid (may involve additional levels)}
\State $\mathbf{u}_k \gets \mathbf{u}_k + {\color{burgundy}\mathbf{P}}_{\text{ctof}} \mathbf{e}$                                         \Comment{prolong error correction}
\State $\mathbf{u}_k \gets \mathbf{u}_k + \hat{\mathbf{M}}^{-1} \left( \mathbf{b} - {\color{burgundy}\mathbf{A}}_f \mathbf{u}_k \right)$ \Comment{post-smooth $\nu$ times}
\end{algorithmic}
\end{definition}
where ${\color{burgundy}\mathbf{A}}_f$ is the operator on the fine grid, ${\color{burgundy}\mathbf{P}}_{\text{ctof}}$ is the coarse to fine grid prolongation operator, ${\color{burgundy}\mathbf{R}}_{\text{ftoc}}$ the fine to coarse grid restriction operator, $\hat{\mathbf{M}}$ a separate preconditioner used for smoothing, and ${\color{burgundy}\mathbf{A}}_c$ represents solving the error correction problem on the coarse grid, which may involve recursively applying the multigrid algorithm.

For $h$-multigrid, the restriction operator represents agregating multiple fine grid elements into larger coarse grid elements and the prolongation operator is given by the transpose.
This process requires careful knowledge about the mesh and can be complex on fully unstructured meshes.

For $p$-multigrid with nodal bases, the prolongation operator interpolates using the lower order basis functions to the higher order basis nodes and is defined by
\begin{equation}
\mathbf{P}_{p - 1}^p = \Lambda \left( m_p^{-1} \right) \mathcal{E}_p^T \sum_e \mathbf{N}_{p - 1}^p \mathcal{E}^e_{p - 1}
\label{mg_prolong}
\end{equation}
where $\mathbf{N}_{p - 1}^p$ interpolates from a basis of degree $p - 1$ to degree $p$ and $m_p = \mathcal{E}_p^T \mathcal{E}_p 1$ counts the multiplicity of shared nodes between elements.
To preserve symmetry and prevent aliasing, the restriction operator is defined as the transpose of the prolongation operator, $\mathbf{R}_{p - 1}^p = \left( \mathbf{P}_{p - 1}^p \right)^T$.
These operators can be implemented in a matrix-free fashion.

Even modest order finite elements, such as degree $4$, $p$-multigrid can substantially reduce the size of the global solution vector by a factor of approximately $p^3 / 8$.
Thus, assembly of the finite element operator on the coarse grid, ${\color{burgundy}\mathbf{A}}_c$, becomes tractable and direct solvers can be used to solve the coarse problem.
