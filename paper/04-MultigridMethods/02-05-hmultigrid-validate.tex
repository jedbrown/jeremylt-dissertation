By representing the fine grid with macro-elements and the prolongation operator with this interpolation, this LFA of p-multigrid exactly reproduces the results of He and Maclachlan \cite{he2020two} for LFA of high-order h-multigrid even though this previous work did not use the concept of macro-elements.
The results from this LFA can be seen in Table \ref{table:two_grid_hmultigrid}.

\begin{table}[ht!]
\begin{center}
\begin{tabular}{l cc cc cc}
  \toprule
  $p, d$  &  \multicolumn{2}{c}{$\nu = \left( 0, 1 \right)$}  &  \multicolumn{2}{c}{$\nu = \left( 1, 1 \right)$}  &  \multicolumn{2}{c}{$\nu = \left( 2, 2 \right)$}  \\
  %\cmidrule(lr){2-3} \cmidrule(lr){4-5} \cmidrule(lr){6-7}
                      &  $\rho$  &  $\omega$  &  $\rho$ & $\omega$  &  $\rho$ & $\omega$  \\
  \toprule
  $p = 2, d = 1$  &  0.821 & 1.000  &  0.821 & 1.000  &  1.279 & 1.000   \\
  $p = 2, d = 1$  &  0.526 & 0.838  &  0.495 & 0.838  &  0.302 & 0.838   \\
  $p = 2, d = 1$  &  0.291 & 0.709  &  0.249 & 0.709  &  0.064 & 0.709   \\
  \midrule
  $p = 3, d = 1$  &  0.491 & 0.650  &  0.337 & 0.650  &  0.131 & 0.650   \\
  \midrule
  $p = 4, d = 1$  &  0.608 & 0.640  &  0.559 & 0.640  &  0.331 & 0.640   \\
  \midrule
  $p = 2, d = 2$  &  0.452 & 1.000  &  0.288 & 1.000  &  0.091 & 1.000   \\
  \bottomrule
\end{tabular}
\end{center}
\caption{Two-grid convergence factor and Jacobi smoothing parameter for high-order h-multigrid}
\label{table:two_grid_hmultigrid}
\end{table}

On uniform rectangular meshes, linear finite elements produce the same discetized operator as finite differencing.
The nine-point stencil for the Laplace operator in 2D is given by

\begin{equation}
\frac{1}{3}
\begin{bmatrix}
-1  &  -1  &  -1   \\
-1  &   8  &  -1   \\
-1  &  -1  &  -1  \\
\end{bmatrix}
\end{equation}
with a corresponding local Fourier analysis symbol given by

\begin{equation}
\tilde{A} \left( \theta_1, \theta_2 \right) = \frac{8}{3} - \frac{2}{3} \cos \left( \theta_1 \right) - \frac{2}{3} \cos \left( \theta_2 \right) - \frac{4}{3} \cos \left( \theta_1 \right) \cos \left( \theta_2 \right)
\end{equation}

The assembled matrix for a single linear element is given by

\begin{equation}
{\color{burgundy}\mathbf{A}}^e =
\frac{1}{3}
\begin{bmatrix}
 2    &  -1/2  &  -1/2  &  -1    \\
-1/2  &   2    &  -1    &  -1/2  \\
-1/2  &  -1    &   2    &  -1/2  \\
-1    &  -1/2  &  -1/2  &   2    \\
\end{bmatrix}
\end{equation}
with a corresponding local Fourier analysis symbol given by

\begin{equation}
\begin{split}
\tilde{\color{burgundy}\mathbf{A}} \left( \theta_1, \theta_2 \right) & = \mathbf{Q}^T \left( {\color{burgundy}\mathbf{A}}^e \odot \left[ e^{\imath \left( \mathbf{x}_j - \mathbf{x}_i \right) \cdot \boldsymbol{\theta}} \right] \right) \mathbf{Q}\\ & = \frac{8}{3} - \frac{2}{3} \cos \left( \theta_1 \right) - \frac{2}{3} \cos \left( \theta_2 \right) - \frac{4}{3} \cos \left( \theta_1 \right) \cos \left( \theta_2 \right).
\end{split}
\end{equation}

We can use the LFA of multigrid given by Definition \ref{def:multigrid_symbol} with the symbols of the h-multigrid transfer operators givn by Definition \ref{def:h_prolongation_symbol} and Definition \ref{def:h_restriction_symbol} to reproduce LFA of h-multigrid methods for finite differencing where the stencil can be represented by a finite element discretization.
This LFA of arbitrary second-order PDEs with high-order finite element discretizations agrees with previous work on LFA of PDE operators derived with finite differencing with analogus stencils.

Our LFA of h-multigrid presented here is based on a specific basis of the Fourier space used in \cite{kumar2019local} rather than the commonly used Fourier modes, see \cite{MR1807961,wienands2004practical}, where the symbol of each component in multigrid methods are formed based on different harmonic frequencies.
The symbols of grid-transfer operators described in Definition \ref{def:h_prolongation_symbol} and Definition \ref{def:h_restriction_symbol} give a general way for multigrid coarsening with factor $m$, and this framework is simpler, especially for high-order discretizations described in \cite{he2020two}.
Also, our LFA for h-multigrid is suitable for finite element discretizations without requiring bases to have uniformly spaced nodes.

As mentioned before, our focus of this work is p-multigrid, so we will not expand the discussion of h-multigrid further method here.
Applying this LFA framework to h-multigrid or hp-multigrid methods are topics for future research.
