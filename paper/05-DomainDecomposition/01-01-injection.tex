The global subassembly injection operator, $\mathbf{R}$, maps the global degrees of freedom to the broken spaces of the subassembled problem on each subdomain.
We consider the two injection operators discussed by Brown, He and MacLachlan in \cite{brown2019local}.

The first injection operator, $\mathbf{R}_1$, is the scaled injection operator, which results in the lumped BDDC preconditioner, $\mathbf{M}^{-1}_1$.

Let $\mathcal{N} \left( x \right)$ denote the set of subdomains that contain a broken copy of degree of freedom $x$ in the broken space.
If $x \in \Pi^e$ or $x \in I^e$, then $\lvert \mathcal{N} \left( x \right) \rvert = 1$, otherwise $\lvert \mathcal{N} \left( x \right) \rvert$ is the number of subdomains that contain $x$.
The scaled injection operator, $\mathbf{R}_1$ is defined as the mapping between the global degrees of freedom and the broken space where each element is scaled by $1 / \lvert \mathcal{N} \left( x \right) \rvert$.
For interior and vertex degrees of freedom, the sparse matrix representation of $\mathbf{R}_1$ would contain a single entry with a value of $1$, while each column corresponding to a duplicated degree of freedom has $\lvert \mathcal{N} \left( x \right) \rvert$ entries, each given by $1 / \lvert \mathcal{N} \left( x \right) \rvert$.

\begin{definition}
Let the scaled injection operator, $\mathbf{R}_1$, be given by the mapping from the global problem space to the broken space where each degree of freedom is scaled by its multiplicity in the broken space, denoted by $\lvert \mathcal{N} \left( x \right) \rvert$.
The lumped BDDC preconditioner is given by $\mathbf{M}^{-1}_1 = \mathbf{R}_1^T \hat{\color{burgundy}\mathbf{A}} \mathbf{R}_1$, where $\mathbf{R}_1$ is the scaled injection operator.
\label{def:lumpedbddc}
\end{definition}

An operator preconditioned with the lumped BDDC preconditioner has the same eigenvalues as the FETI-DP preconditioned operator, with the exception of some eigenvalues equal to $0$ and $1$ \cite{li2007use}.

The second injection operator, $\mathbf{R}_2$, is the harmonic injection operator, which results in the Dirichlet BDDC preconditioner, $\mathbf{M}^{-1}_2$.

This injection operator uses a discrete harmonic extension of the interface values to minimize the energy norm of the result, which leads to a better stability bound, as shown by Toselli and Widlund \cite{toselli2006domain}.
Let $\mathcal{H}$ be given by a direct sum of local operators $\mathcal{H}^e = - {\color{burgundy}\mathbf{A}}_{\text{I}, \text{I}}^{e, -1} {\color{burgundy}\mathbf{A}}_{\Gamma, \text{I}}^{e, T}$.
This operator solves a local Dirichlet problem given by mapping the jump over subdomain interfaces for duplicated values to the subdomain interior.
The jump mapping is given by
\begin{equation}
\mathbf{J}^{e, T} v \left( x \right) = \sum_{i \in \mathcal{N} \left( x \right)} \left( \frac{v^e \left( x \right)}{\lvert \mathcal{N} \left( x \right) \rvert} - \frac{v^i \left( x \right)}{\lvert \mathcal{N} \left( x \right) \rvert} \right), \forall x \in \Gamma^e.
\end{equation}

\begin{definition}
Let the harmonic injection operator be given by $\mathbf{R}_2 = \mathbf{R}_1 - \mathbf{J}^T \mathcal{H}^T$, where $\mathcal{H}$ is given by a direct sum of local operators $\mathcal{H}^e = - {\color{burgundy}\mathbf{A}}_{\text{I}, \text{I}}^{e, -1} {\color{burgundy}\mathbf{A}}_{\Gamma, \text{I}}^{e, T}$ and
\begin{equation}
\mathbf{J}^{e, T} v \left( x \right) = \sum_{i \in \mathcal{N} \left( x \right)} \left( \frac{v^e \left( x \right)}{\lvert \mathcal{N} \left( x \right) \rvert} - \frac{v^i \left( x \right)}{\lvert \mathcal{N} \left( x \right) \rvert} \right), \forall x \in \Gamma^e.
\end{equation}
The lumped BDDC preconditioner is given by $\mathbf{M}^{-1}_2 = \mathbf{R}_2^T \hat{\color{burgundy}\mathbf{A}} \mathbf{R}_2$, where $\mathbf{R}_2$ is the harmonic injection operator.
\label{def:dirichletbddc}
\end{definition}

An operator preconditioned with the Dirichlet BDDC preconditioner, $\mathbf{M}^{-1}_2 {\color{burgundy}\mathbf{A}}$, has the same eigenvalues as the BDDC preconditioned operator with original formulation of the BDDC preconditioner given by Dohrmann \cite{dohrmann2003preconditioner}, with the exception of some eigenvalues that are equal to $1$ \cite{li2007use}.
