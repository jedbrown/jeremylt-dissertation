Balancing Domain Decomposition by Constraints (BDDC) is a relatively new technique created by Dohrmann in 2003 \cite{dohrmann2003preconditioner}.
BDDC is an improvement upon Balancing Domain Decomposition (BDD) \cite{mandel1993balancing}, which requires modifications to be effective for high-order problems, especially for three-dimensional problems.
The dual primal version of finite element tearing and interconnecting (FETI-DP) \cite{farhat2000scalable} is closely related to BDDC and the two are effectively the same technique, under certain assumptions \cite{mandel2007bddc}.

Domain decomposition techniques decompose the domain $\Omega$ into a series of subdomains $\Omega^i$.
Some techniques use overlapping subdomains, such as additive Schwartz methods, which are popular for high-order and spectral finite elements \cite{fischer1997overlapping} and have been used as smoothers in multigrid methods \cite{fischer2005hybrid}.
In BDDC, we instead use non-overlapping subdomains with an energy minimization problem to resolve jumps in the values of degrees of freedom over subdomain interfaces.

LFA of BDDC was introduced by Brown, He, and MacLachlan \cite{brown2019local}.
We will discuss implementing BDDC in a matrix-free fashion and re-derive the LFA of BDDC in the context of high-order elements, with single element and macro-element subdomains.
