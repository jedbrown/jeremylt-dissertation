High-order finite elements implemented in a matrix-free fashion are one way to address the memory bandwidth bottlenecks for modern HPC hardware.
In this chapter, we consider the performance constraints of modern HPC hardware and present a representation of high-order finite elements that facilitates matrix-free implementation.

In Section \ref{sec:highorderwhy} we explore the limitations of modern HPC hardware and in Section \ref{sec:highorderbenefits} we discuss the benefits of high-order matrix-free methods on modern HPC hardware.
In Section \ref{sec:highordernotation} we develop notation to describe high-order finite element discretizations for arbitrary second order PDEs implemented in a matrix-free fashion on unstructured meshes.
In Section \ref{sec:highordersolvers} we discuss iterative solvers for matrix-free methods as well as the importance of preconditioning high-order finite element operators for the convergence of these solvers.
Finally, in Section \ref{sec:libceedapplications} we provide some examples of practical applications with libCEED that demonstrate the flexibility and applicability of this matrix-free representation for real-world problems.
