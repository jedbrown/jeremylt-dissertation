High-order finite elements offer high accuracy and exponential convergence.
The spectral convergence properties of high order finite elements have been extensively discussed, such as in \cite{guo1986hp}.
In general, the discretization error in the finite element approximation is given by
\begin{equation}
E_{H^1 \left( \Omega \right)} \leq C \left( p \right) h^{\text{min} \left( k, p + 1 \right)}
\label{eq:spectral_convergence}
\end{equation}
where $p$ is the polynomial order of the mesh, $h$ is the size of the finite elements, and $k$ is the order of the Sobolev space to which the true solution belongs.
The coefficient $C$ does depend upon the polynomial order of the finite element basis, but in practice the exponential term dominates this error approximation.

EXPAND SPECTRAL ACCURACY AND LIMITATIONS THEREIN

In practical problems, the smoothness of the solution may prevent spectral convergence from being achieved; however, high-order finite elements will still offer convergence that is no worse than the convergence on a comparable low-order mesh with a larger number of elements.
In those cases, high-order finite elements implemented in a matrix-free fashion still offer the storage and FLOPs benefits detailed above in Section \ref{sec:storageandflops}.
The convergence of high-order methods have been extensively investigated; for further discussion, see \cite{babuska1982rates}, among others.

Demkowicz, Oden, and Rachowicz, et al. discussed $hp$ adaptivity in the context of minimizing the total number of degrees of freedom (DoFs) required to achieve a target accuracy \cite{demkowicz1989toward}, \cite{oden1989toward}, \cite{rachowicz1989toward}.
However, in $hp$ adaptivity, the polynomial order of each finite element may differ.
While problems utilizing these types of discretizations can be implemented in a matrix-free fashion with the preconditioners discussed in Chapter \ref{ch:LocalFourierAnalysis}, we instead focus on meshes where all elements have the same polynomial order in order to simplify the analysis and implementation details.
