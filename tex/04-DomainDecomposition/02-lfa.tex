In order to compute the symbol of the BDDC preconditioners, we will find the symbol of the injection operator and subassembled inverse separately.

The inverse of the subassembled problem requires both modes in the broken element space and the global primal variable space.

The Fourier modes on each broken element subdomain are separate, so the mode mapping operator $\mathbf{Q}$ is not used for the symbol of the subdomain solver.
The subdomain solver is completely in the broken element space therefore has a very straightforward symbol, given by
\begin{equation}
\tilde{\color{burgundy}\mathbf{A}}_{\text{r}, \text{r}}^{-1} \left( \boldsymbol{\theta} \right) = {\color{burgundy}\mathbf{A}}_{\text{r}, \text{r}}^{-1} \odot \left[ e^{\imath \left( \mathbf{x}_j - \mathbf{x}_i \right) \cdot \boldsymbol{\theta} / \mathbf{h}} \right]
\end{equation}
where $\mathbf{x}_i$ and $\mathbf{x}_j$ are only the coordinates for the nodes in the subassembled space.

The Schur complement has as similarly straightforward symbol, given by
\begin{equation}
\tilde{\hat{\mathbf{S}}}_{\Pi}^{-1} \left( \boldsymbol{\theta} \right) = \left( \mathbf{Q}^T \left( \hat{\mathbf{S}}_{\Pi} \odot \left[ e^{\imath \left( \mathbf{x}_j - \mathbf{x}_i \right) \cdot \boldsymbol{\theta} / \mathbf{h}} \right] \right) \mathbf{Q} \right)^{-1}.
\end{equation}
The primal vertex solve is global across the entire problem, so the corresponding Fourier modes are mapped to the global problem before being solved exactly.

For the mixed primal and subdomain operators, we use the Fourier mode mapping only on the modes corresponding to the primal vertices, giving us
\begin{equation}
\tilde{\hat{\color{burgundy}\mathbf{A}}}\color{black}_{\text{r}, \Pi} \left( \boldsymbol{\theta} \right) = \left( \hat{\color{burgundy}\mathbf{A}}_{\text{r}, \Pi} \odot \left[ e^{\imath \left( \mathbf{x}_j - \mathbf{x}_i \right) \cdot \boldsymbol{\theta} / \mathbf{h}} \right] \right) \mathbf{Q}
\end{equation}
where $x_i$ come from the broken element space and $x_j$ come from the primal variable space.

\begin{definition}
The symbol of the inverse of the subassembled problem is given by
\begin{equation}
\tilde{\hat{\color{burgundy} \mathbf{A}}}\color{black}^{-1} =
\left[ \begin{array}{c c}
\mathbf{I}  &  -\tilde{\color{burgundy}\mathbf{A}}_{\text{r}, \text{r}}^{-1} \tilde{\hat{\color{burgundy}\mathbf{A}}}\color{black}_{\Pi, \text{r}}^T  \\
\mathbf{0}  &  \mathbf{I}                                                                                                                             \\
\end{array} \right]
\left[ \begin{array}{c c}
\tilde{\color{burgundy}\mathbf{A}}_{\text{r}, \text{r}}^{-1}  &  \mathbf{0}                                        \\
\mathbf{0}                                                    &  \tilde{\hat{\mathbf{S}}}\color{black}_{\Pi}^{-1}  \\
\end{array} \right]
\left[ \begin{array}{c c}
\mathbf{I}                                                                                                                           &  \mathbf{0}  \\
-\tilde{\color{burgundy}\mathbf{A}}_{\Pi, \text{r}} \tilde{\hat{\color{burgundy}\mathbf{A}}}\color{black}_{\text{r}, \text{r}}^{-1}  &  \mathbf{I}  \\
\end{array} \right]
\end{equation}
In this notation, $\tilde{\color{burgundy}\mathbf{A}}$ indicates Fourier modes on the broken element space, while $\tilde{\hat{\color{burgundy}\mathbf{A}}}$ indicates Fourier modes on the global primal variable space.

The symbol of the subdomain solver is given by
\begin{equation}
\tilde{\color{burgundy}\mathbf{A}}_{\text{r}, \text{r}}^{-1} \left( \boldsymbol{\theta} \right) = {\color{burgundy}\mathbf{A}}_{\text{r}, \text{r}}^{-1} \odot \left[ e^{\imath \left( \mathbf{x}_j - \mathbf{x}_i \right) \cdot \boldsymbol{\theta} / \mathbf{h}} \right]
\end{equation}
while the symbol of the Schur compliment solve is given by
\begin{equation}
\tilde{\hat{\mathbf{S}}}_{\Pi}^{-1} \left( \boldsymbol{\theta} \right) = \left( \mathbf{Q}^T \left( \hat{\mathbf{S}}_{\Pi} \odot \left[ e^{\imath \left( \mathbf{x}_j - \mathbf{x}_i \right) \cdot \boldsymbol{\theta} / \mathbf{h}} \right] \right) \mathbf{Q} \right)^{-1}
\end{equation}
and the symbol of the mixed primal and subdomain operators are given by
\begin{equation}
\tilde{\hat{\color{burgundy}\mathbf{A}}}\color{black}_{\text{r}, \Pi} \left( \boldsymbol{\theta} \right) = \left( \hat{\color{burgundy}\mathbf{A}}_{\text{r}, \Pi} \odot \left[ e^{\imath \left( \mathbf{x}_j - \mathbf{x}_i \right) \cdot \boldsymbol{\theta} / \mathbf{h}} \right] \right) \mathbf{Q}
\end{equation}
and $\tilde{\hat{\color{burgundy}\mathbf{A}}}\color{black}_{\text{r}, \Pi}^T \left( \boldsymbol{\theta} \right)$ is given by the conjugate transpose.
\label{def:subassembled_symbol}
\end{definition}
